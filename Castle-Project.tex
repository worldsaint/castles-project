\documentclass{article}

\begin{document}
\title{Jogo de estratégia}
\maketitle
\newpage
\section*{Criando Castelos}
Veja o diagrama abaixo. Ele representa um sistema simples para representação de diversos castelos para um jogo de estratégia medieval. Nesse diagrama '#' representa visibilidade protegida(protected) e  '+' representa público (public).
\par A tabela abaixo descreve como funcionam os métodos descritos nas classes.
Escreva código de cada uma das classes desse diagrama usando a linguagem JAVA. \\
\begin{tabular}{|c|c|c|}
	void ataque(double d) & Castelo & Se o campo defesas for > 0.0, então, reduzir o valor deste campo, senão reduzir o valor do campo​ pontosVida. \\ 		& Europeu & Se​ o número​ de​Paladinos ou​​ Samurais​ for > 0 , realizar um sorteio​ aleatório e​ eliminar entre​2​ e​ 5​ guerreiros.​ Senão,​proceder com o método ataque da​ classe Castelo. \\
	& Japones & \\
	void mostrarSituacao() & & Mostrar​ os valores​ contidos nos campos da classe. Se​ pontosVida​ == 0.0, então, mostrar​ também​ a mensagem o​ castelo foi​ destruído. \\
\end{tabular}
\par Crie classes de testes​ para cada uma das classes acima, incluindo testes​ positivos​ e negativos.
\section*{Criando personagens}
De maneira similar aos castelos, crie uma hierarquia de classes para
representar personagens do jogador no universo do Jogo. Nosso jogo permite ao jogador atuar como paladino ou samurai.​ Cada​ personagem​ deve​ conter pelo​menos​ os​ seguintes atributos:
\begin{itemize}
\item HP​ : Hit Points determinam quanto dano o personagem pode receber antes de morrer. O HP pode ser recuperado através de magias, mas caso chegue a zero, o personagem será retirado do mapa e​ não poderá recuperá-los.
\item Força​ : É a característica que determina quanto dano o personagem causa em um ataque físico.​ Cada personagem​ possui​​ uma força baseada em​ suas características físicas.
\item Defesa​ : Defesa é a característica responsável pela diminuição do dano recebido durante um ataque físico. Um​ ataque​ físico sempre causará, pelo​ menos,​um​ ponto de dano.
\item Crítico​ : Essa característica determina a chance (em porcentagem) do
personagem causar dano extra ao realizar​ um​ ataque físico.
\end{itemize}
Personagens podem se movimentar pelo mapa do jogo, atacar uns aos outros, ou atacar castelos inimigos,​desde​ que​​ estejam​ em uma posição​ adjacente ao​seu alvo.

\section*{Mapeando o​ jogo}
O universo do jogo é representado através de uma matriz N x N, onde cada célula da matriz representa uma posição. Adicione uma classe para representar o mapa do jogo, incluindo métodos que permitam acessar/setar o  valor de uma determinada​​ posição.
\par Cada posição pode ser ocupada por um castelo ou por um personagem. Personagens são os únicos que podem se mexer. Você precisará adicionar uma classe pai á sua hierarquia de modo que todos os elementos do jogo possam estar representados no mapa. Explore o polimorfismo de métodos para implementar a​ funcionalidade​ de movimentação.
\section*{As mecânicas}
A classe principal de qualquer game é a classe que contém o gameloop. A
classe Game não será responsável por um loop e sim por gerir o nosso jogo com os métodos que você precisa criar. Essa classe será responsável, também, por armazenar as coleções e será ela quem irá criar um novo objeto quando necessário através de algum método gerador. 
\par Dicas: métodos como mostrar tabuleiro, mostrar recursos, numero de soldados, etc... Pense nisso como o menu principal do seu jogo.
\par Nosso jogo funcionará da seguinte maneira: Existirão dois castelos. Castelos serão representados na matriz tabuleiro como “E” para castelo Europeu e “J” para castelo japonês. Ambos devem ficar em posições distintas do mapa, só podendo ficar separadas através de, pelo menos, 3
casas em qualquer direção (diagonais não medíveis, ou seja, utilize linhas ou colunas). Os jogadores começarão com personagens randomicamente gerados pelo mapa. Cada nação começará com 3 “soldados” de cada lado. O controle será feito por linha de comando (procure na API por java scanner). Defina os botões de movimento como você preferir. Quando um soldado inimigo passar por cima do seu, o movimento não deve ser considerado, entretanto um ataque foi iniciado. Isso significa que, caso exista um personagem no endereço (3,3) dá matriz e um personagem em (4,3), caso o personagem​ de cima ande​ para​ baixo​ ou vice​ versa, um ataque deverá​ ser iniciado.
\section*{Resolução de combates}
Para resolver um combate, precisamos nos ater aos atributos e aos estados dos personagens. Quando ocorrer, o jogador atacante irá lançar um dado d6 (devendo gerar um número aleatório entre 1 e 6). Caso o número seja menor ou igual à 2, o jogador vai ter errado o ataque e nada acontecerá. Caso o valor tenha sido 3, 4 ou 5, o jogador acertará um dano de acordo com a
defesa do inimigo, isto é: (Valor do dado * valor de ataque - defesa do
inimigo). Portanto, se temos 10 de ataque e o inimigo 20 de defesa, tirando 5 nos dados teríamos: (5 * 10 - 20 = 30) de dano.
\par Reduzimos este valor dos pontos de vida do inimigo. Caso o inimigo fique sem pontos de vida, ele deverá ser removido da matriz tabuleiro. Caso o valor do dado seja 6, você causará piercing na armadura inimiga, portanto, multiplicará o dano por 3 e somará 15 ao valor do dano. Nesse caso, a armadura inimiga​ será descartada​  e​ o​ dano​ será​ todo projetado.
\par Exemplo de movimentação em um tabuleiro 4 por 4 (E = castelo europeu, J = castelo japonês,​ S​ =​ Samurai, P =​ Paladino).
\begin{tabular}{c c c c}
	E & 0 &​ 0 &​ P \\
	0 &​ 0 &​ 0 &​ 0 \\
	0 &​ 0 &​ S &​ 0 \\
	0​ & J & 0 &​ 0
\end{tabular}
\par É a​ vez​ do paladino,​ o​ jogador pressiona​ ‘s’​​(indicando que está se movendo para baixo) e dá enter:
\begin{tabular}{c c c c}
	E & 0 & 0 & 0 \\
	0 & 0 & 0 & \textbf{P} \\
	0 & 0 & S & 0 \\
	0 & J & 0 & 0
\end{tabular}
\par É a​ vez​ do samurai,​ o​ jogador​ pressiona​ ‘w’​(indicando que está se movendo para cima) e dá enter:
\begin{tabular}{c c c c}
	E & 0 & 0 & 0 \\
	0 & 0 & \textbf{S} & P \\
	0 & 0 & 0 & 0 \\
	0 & J & 0 & 0
\end{tabular}
\par Agora, caso o jogador que controla o paladino se mova para a esquerda, ele deverá entrar em combate.
\par A cada 5 movimentos, um jogador do seu time deve nascer randomicamente
adjacente ao castelo​de​ sua nação.​ Isso vai gerar continuidade​ ao jogo.
\section*{Vitória}
Para vencer o jogo, o seu time deverá destruir o castelo inimigo. Os soldados devem sumir ao se mover em direção ao castelo inimigo, mas devem causar dano à estrutura de modo que 5 guerreiros sejam o​ suficiente​​ para derrubá-lo.
\subsubsection*{Lembre-se}
\par Estamos falando de um jogo: a liberdade criativa deve lhe guiar. Modifique as mecânicas e balanceie seu jogo como preferir. Veja que não falamos sobre seletor de jogador, como você vai saber qual jogar vai se movimentar nesta rodada? Fica à seu critério. Você pode fazer isso de maneira aleatória, de maneira sequencial ou até mesmo escolher qual de seus
guerreiros deverá movimentar. Queremos ver sua capacidade de criar classes com baixo acoplamento (baixa dependência de outras classes), gerir coleções e utilizar herança e polimorfismo, portanto, as mecânicas são secundárias. Tente implementar o máximo que você viu em sala para fazer o jogo funcionar.

\end{document}
